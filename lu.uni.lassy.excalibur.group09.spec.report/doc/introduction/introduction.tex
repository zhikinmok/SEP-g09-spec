% Last Modification:
% @author AUTHOR_NAME
% @date TODAY_DATE

\chapter{Introduction}
\label{chap:introduction}

\section{Overview}
HighwayToSafety is a simple system intended to help users organising and
managing the aftermath of an highway car crash as effectively as possible. It
assures a quick communication between the workforce, who are on the site of the
accident and their colleagues who work at the emergency centrals.

\section{Purpose and recipients of the document}
This document is an analysis document complying with the \msrcode{MESSIR}
methodology. Its intent is to provide a first simple example of a precise
specification of the functional properties of the \msrcode{HTS} system.
\newline

The recipients of this document are:

\begin{itemize}

\item the \msrcode{HTS} system's buyer company (ABC): this document is used as a
contractual document jointly with any other document considered as useful (as
requirement elicitation document, \ldots) in order to have a higher degree of
precision in requirement description. It is also used as a basis document
for the \msrcode{HTS} system validation using specification based testing.
\item the \msrcode{HTS} system development company (ADC) will use this document as
the basis for development (mainly design, implementation, maintenance). It is
also used for verification and validation using test plans defined using the
analysis models described in this document and according to the \msrcode{HTS} methodology.
 
\end{itemize} 
 
\section{Application Domain}

The \msrcode{HTS} system belongs to the Crisis Management Systems Domain. It is a
system dedicated to only crisis professional end users.

It is not an institutional system certified and guaranteed by any governmental
entity and thus must be used with caution. 
\section{Definitions, acronyms and abbreviations}
\msrcode{HTS} - HighwayToSafety

\section{Document structure} 
The document structure is designed to be coherent with the
\msrmessirmeth~\cite{messirbook}. Section \ref{chap:general_description} provides a general
description of the system purpose, its users, its environment and some general
non functional requirements. A more detailed description of the non functional
requirements related to dependability, if any, are provided in section 7.
The \glspl{systop} triggered by input events coming from the external
\glspl{actor} belonging to the environment are described in Section 3.
The \msrcode{HTS} concepts used to represent the any persistent or transient
information is given in Section 4. The precise specification of the system
operations in term of system's state changes, events sent together with the
constraints on the allowed sequences of system operations are described in
Section 5.
